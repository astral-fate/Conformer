\begin{abstract}
The field of Continuous Sign Language Recognition (CSLR) presents a substantial technical challenges including fluid inter sign transitions, absence of temporal boundaries, and co-articulation effects. This paper, conducted for the MSLR 2025 Workshop Challenge at ICCV 2025, addresses signer-independent recognition challenges to develop robust CSLR systems that generalize across diverse signers and improve Arabic sign language recognition. A comprehensive data-centric methodology was developed encompassing systematic feature engineering, preprocessing pipeline design, and model architecture optimization on the Isharah dataset. Primary contributions include: (1) systematic feature engineering through EDA, identifying communicative keypoints while reducing body region noise; (2) robust pre-processing with DBSCAN-based outlier filtering and spatial normalization; (3) novel CSLRConformer architecture adapting Conformer's hybrid CNN-Transformer design for sign language, combining convolutional layers for local temporal dependencies with self-attention for global sequence modeling. The methodology achieved competitive performance with 5.60\% Word Error Rate on development set and 12.70\% on the test set, demonstrating 75.1\% relative WER reduction on development set and 53.6\% on test set compared to the best-performing baselines from the original Isharah dataset, and establishing that systematic, data-driven feature engineering coupled with hybrid architectural design provides substantial performance gains for CSLR applications.

\textbf{\textit{Keywords}}: Continuous Sign Language Recognition (CSLR),  Hybrid CNN-Transformer, CSLRConformer



\end{abstract}