\section{{Conclusion and Future Work}} 
\label{sec:formatting}
This paper presents a comprehensive data-centric methodology for Continuous Sign Language Recognition, specifically developed for the MSLR 2025 Workshop Challenge using the Isharah dataset. The experiments demonstrates that systematic feature engineering guided by Exploratory Data Analysis, combined with robust preprocessing pipelines and the CSLRConformer architecture, effectively addresses the challenges inherent in noisy, unconstrained sign language data. The final model achieved a competitive Word Error Rate of 12.70\% on the test set, with ablation studies confirming that data-driven feature selection contributed to a substantial 45\% relative improvement in performance compared to baseline approaches. 

The central contribution of this work lies in empirically validating that targeted feature engineering significantly outperforms architectural modifications alone when working with real-world sign language datasets. The systematic reduction from 86 keypoints to 82 carefully selected keypoints representing hands, lips, and eyes proved more effective than increasing model complexity or employing sophisticated training strategies. This finding aligns with emerging evidence in the field emphasizing the critical importance of domain-informed pre-processing over purely model-centric approaches \cite{preprocessing_keypoint_sign}. 

Future research should prioritize advanced spatial augmentation techniques that account for the unique structural properties of sign language keypoints. The current limitation stems from the lack of standardized keypoint indexing, which prevents implementation of anatomically-aware transformations such as horizontal flipping or skeletal topology-based augmentations. Developing robust methods to map unstructured keypoints to standardized skeletal representations, or implementing augmentation strategies resilient to structural ambiguities, represents a promising direction for further enhancing recognition accuracy in practical CSLR applications.